% !TeX document-id = {a85810fc-bdb6-4afa-ac04-1ff122189b0a}
% !TeX root = ./简历-殷政浩-京都大学.tex
% !TEX program = xelatex
% !BIB program = biber
% !TEX encoding = UTF-8 Unicode
% !TEX options = --shell-escape -synctex=1 -interaction=nonstopmode -file-line-error "%DOC%"
%%%%%%%%%%%%%%%%%
% This is an example CV created using altacv.cls (v1.1.4, 27 July 2018) written by
% LianTze Lim (liantze@gmail.com), based on the
% Cv created by BusinessInsider at http://www.businessinsider.my/a-sample-resume-for-marissa-mayer-2016-7/?r=US&IR=T
%
%% It may be distributed and/or modified under the
%% conditions of the LaTeX Project Public License, either version 1.3
%% of this license or (at your option) any later version.
%% The latest version of this license is in
%%    http://www.latex-project.org/lppl.txt
%% and version 1.3 or later is part of all distributions of LaTeX
%% version 2003/12/01 or later.
%%%%%%%%%%%%%%%%

%% If you want to use \orcid or the
%% academicons icons, add "academicons"
%% to the \documentclass options.
%% Then compile with XeLaTeX or LuaLaTeX.
% \documentclass[10pt,a4paper,academicons]{altacv}

%% Use the "normalphoto" option if you want a normal photo instead of cropped to a circle
% \documentclass[10pt,a4paper,normalphoto]{altacv}

\documentclass[10pt,a4paper]{altacv}

%% AltaCV uses the fontawesome and academicon fonts
%% and packages.
%% See texdoc.net/pkg/fontawecome and http://texdoc.net/pkg/academicons for full list of symbols.
%% When using the "academicons" option,
%% Compile with LuaLaTeX for best results. If you
%% want to use XeLaTeX, you may need to install
%% Academicons.ttf in your operating system's font %% folder.


% Change the page layout if you need to
\geometry{left=1cm,right=9cm,marginparwidth=6.8cm,marginparsep=1.2cm,top=1cm,bottom=1cm}

% Change the font if you want to.

% If using pdflatex:
\usepackage[utf8]{inputenc}
\usepackage[T1]{fontenc}
\usepackage[default]{lato}
\usepackage{hyperref}
\usepackage{xeCJK}
\setCJKmainfont{Source Han Sans SC}

% If using xelatex or lualatex:
\setmainfont{Lato}

% Change the colours if you want to
\definecolor{Blau}{HTML}{4a91c4}
\definecolor{VividPurple}{HTML}{3E0097}
\definecolor{SlateGrey}{HTML}{2E2E2E}
\definecolor{LightGrey}{HTML}{666666}
\colorlet{heading}{Blau}
\colorlet{accent}{Blau}
\colorlet{emphasis}{SlateGrey}
\colorlet{body}{LightGrey}

% Change the bullets for itemize and rating marker
% for \cvskill if you want to
\renewcommand{\itemmarker}{{\small\textbullet}}
\renewcommand{\ratingmarker}{\faCircle}

%% sample.bib contains your publications
\addbibresource{sample.bib}

\begin{document}
\name{殷政浩}
\tagline{}
% Cropped to square from https://en.wikipedia.org/wiki/Marissa_Mayer#/media/File:Marissa_Mayer_May_2014_(cropped).jpg, CC-BY 2.0
% \photo{2.5cm}{square}
\personalinfo{%
	% Not all of these are required!
	% You can add your own with \printinfo{symbol}{detail}
	\mailaddress{\href{mailto:yin.zhenghao@st.kyoto-u.ac.jp}{yin.zhenghao@st.kyoto-u.ac.jp}}
	\phone{(81)080-2383-7348}
	\newline
	\email{A1-211,Kyoto University Katsura, Nishikyo-ku, Kyoto, 615-8510}
	%   \location{Kyoto, JP}
	\newline
	\homepage{\href{https://fibomat.github.io/blog}{https://fibomat.github.io/blog}}
	%   \twitter{@marissamayer}
	%   \linkedin{linkedin.com/in/marissamayer}
	\github{\href{https://github.com/fibomat}{fibomat}} % I'm just making this up though.
	%   \orcid{orcid.org/0000-0000-0000-0000} % Obviously making this up too. If you want to use this field (and also other academicons symbols), add "academicons" option to \documentclass{altacv}
}
	
	%% Make the header extend all the way to the right, if you want.
	\begin{fullwidth}
		\makecvheader
	\end{fullwidth}
	
	%% Depending on your tastes, you may want to make fonts of itemize environments slightly smaller
	\AtBeginEnvironment{itemize}{\small}
	
	%% Provide the file name containing the sidebar contents as an optional parameter to \cvsection.
	%% You can always just use \marginpar{...} if you do
	%% not need to align the top of the contents to any
	%% \cvsection title in the "main" bar.
	\cvsection[page1sidebar_zh]{研究方向}
	
	\cvevent{集成光量子信息处理}{导师~竹内繁树 教授}{2017年 十月 -- 至今}{日本~京都大学}
	在与九州大学横山研究室的合作下,基于氮化硅以及非晶硅的材料平台,我们致力于实现片上高强度宽频带的纠缠光子源。
	\begin{itemize}
		\item 增强型槽形波导的设计与优化
		\item 实现了具有高品质因子的光微环谐振器
		\item 在非线性光学过程中引入电光聚合物
		\item 从化合物沉积到图案形成、刻蚀,对微纳加工有较为全面的了解 
	\end{itemize}
	\divider
	
	\cvevent{量子光学和量子信息}{导师~马小松教授}{2015年 九月 -- 2017年 六月}\newline{中国~南京大学}
	在进入课题组之初,以量子密钥分发和贝尔态产生为研究课题,随后转移到光量子器件的集成化与平台搭建。
	\begin{itemize}
		\item 绝缘体上硅以及绝缘体上氮化硅的微纳加工以及工艺优化
		\item 集成光子学测试平台的实现以及自动化
	\end{itemize}
	\divider
	
	% \cvevent{Electrorheological Fluids}{with Prof. Yonghua Pan and Prof. Huibin Gao}{Nov. 2014 - Oct 2015}{}
	% As a part of undergraduate training program, we evaluated several kinds of EF, and simulated the interaction with electromagnetic fields. Eventually, we succeeded to demonstrate a practical device. \divider
	
	\cvsection{学术交流}
	
	\cvevent{第17届全国量子光学会议}{}
	{2016年八月}{中国~兰州}
	\cvevent{第2届凝聚态物理大会}{}
	{2015年七月}{中国~南京}
	\cvevent{第12届全国高等学校物理演示教学研讨会}{}
	{2015年十月}{中国~苏州}
	\cvevent{2014年浙江大学物理拔尖学生夏令营}{}
	{2014年七月}{中国~杭州}
	
	
	% \cvsection{A Day of My Life}
	
	% % Adapted from @Jake's answer from http://tex.stackexchange.com/a/82729/226
	% % \wheelchart{outer radius}{inner radius}{
	% % comma-separated list of value/text width/color/detail}
	% % Some ad-hoc tweaking to adjust the labels so that they don't overlap
	% \wheelchart{1.5cm}{0.5cm}{%
	%   10/10em/accent!30/Sleeping \& dreaming about work,
	%   25/9em/accent!60/Public resolving issues with Yahoo!\ investors,
	%   5/13em/accent!10/\footnotesize\\[1ex]New York \& San Francisco Ballet Jawbone board member,
	%   20/15em/accent!40/Spending time with family,
	%   5/8em/accent!20/\footnotesize Business development for Yahoo!\ after the Verizon acquisition,
	%   30/9em/accent/Showing Yahoo!\ employees that their work has meaning,
	%   5/8em/accent!20/Baking cupcakes
	% }
	
	
	\clearpage
	
	
	% \cvsection[page2sidebar]{Publications}
	
	
	% \cvsection[page2sidebar]{Publications}
	
	% \nocite{*}
	
	% \printbibliography[heading=pubtype,title={\printinfo{\faBook}{Books}},type=book]
	
	% \divider
	
	% \printbibliography[heading=pubtype,title={\printinfo{\faFileTextO}{Journal Articles}}, type=article]
	
	% \divider
	
	% \printbibliography[heading=pubtype,title={\printinfo{\faGroup}{Conference Proceedings}},type=inproceedings]
	
	% % If the NEXT page doesn't start with a \cvsection but you'd
	% % still like to add a sidebar, then use this command on THIS
	% % page to add it. The optional argument lets you pull up the
	% % sidebar a bit so that it looks aligned with the top of the
	% % main column.
	% \addnextpagesidebar[-1ex]{page3sidebar}
	
\end{document}

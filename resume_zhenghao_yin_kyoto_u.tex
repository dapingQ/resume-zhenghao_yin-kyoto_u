% !TeX document-id = {c3c875d8-04c7-4066-b711-eba665723f62}
% !TeX root = ./resume_zhenghao_yin_kyoto_u.tex
% !TEX program = xelatex
% !BIB program = biber
% !TEX encoding = UTF-8 Unicode
% !TEX options = --shell-escape -synctex=1 -interaction=nonstopmode -file-line-error "%DOC%"
%%%%%%%%%%%%%%%%%
% This is an example CV created using altacv.cls (v1.1.4, 27 July 2018) written by
% LianTze Lim (liantze@gmail.com), based on the
% Cv created by BusinessInsider at http://www.businessinsider.my/a-sample-resume-for-marissa-mayer-2016-7/?r=US&IR=T
%
%% It may be distributed and/or modified under the
%% conditions of the LaTeX Project Public License, either version 1.3
%% of this license or (at your option) any later version.
%% The latest version of this license is in
%%    http://www.latex-project.org/lppl.txt
%% and version 1.3 or later is part of all distributions of LaTeX
%% version 2003/12/01 or later.
%%%%%%%%%%%%%%%%

%% If you want to use \orcid or the
%% academicons icons, add "academicons"
%% to the \documentclass options.
%% Then compile with XeLaTeX or LuaLaTeX.
% \documentclass[10pt,a4paper,academicons]{altacv}

%% Use the "normalphoto" option if you want a normal photo instead of cropped to a circle
% \documentclass[10pt,a4paper,normalphoto]{altacv}

\documentclass[12pt,a4paper]{altacv}

%% AltaCV uses the fontawesome and academicon fonts
%% and packages.
%% See texdoc.net/pkg/fontawecome and http://texdoc.net/pkg/academicons for full list of symbols.
%% When using the "academicons" option,
%% Compile with LuaLaTeX for best results. If you
%% want to use XeLaTeX, you may need to install
%% Academicons.ttf in your operating system's font %% folder.


% Change the page layout if you need to
\geometry{left=2cm,right=9cm,marginparwidth=6.8cm,marginparsep=1.2cm,top=1cm,bottom=1cm}

% Change the font if you want to.

% If using pdflatex:
\usepackage[utf8]{inputenc}
\usepackage[T1]{fontenc}
\usepackage[default]{lato}
\usepackage{hyperref}


% If using xelatex or lualatex:
% \setmainfont{Lato}

% Change the colours if you want to
\definecolor{Blau}{HTML}{4a91c4}
\definecolor{VividPurple}{HTML}{3E0097}
\definecolor{SlateGrey}{HTML}{2E2E2E}
\definecolor{LightGrey}{HTML}{666666}
\colorlet{heading}{Blau}
\colorlet{accent}{Blau}
\colorlet{emphasis}{SlateGrey}
\colorlet{body}{LightGrey}

% Change the bullets for itemize and rating marker
% for \cvskill if you want to
\renewcommand{\itemmarker}{{\small\textbullet}}
\renewcommand{\ratingmarker}{\faCircle}

%% sample.bib contains your publications
\addbibresource{sample.bib}

\begin{document}
\name{Zhenghao Yin}
\tagline{}
% Cropped to square from https://en.wikipedia.org/wiki/Marissa_Mayer#/media/File:Marissa_Mayer_May_2014_(cropped).jpg, CC-BY 2.0
% \photo{2.5cm}{square}
\personalinfo{%
  % Not all of these are required!
  % You can add your own with \printinfo{symbol}{detail}
  \mailaddress{\href{mailto:yin.zhenghao@st.kyoto-u.ac.jp}{yin.zhenghao@st.kyoto-u.ac.jp}}
  \phone{(81)080-2383-7348}
  \newline
  \email{A1-211,Kyoto University Katsura, Nishikyo-ku, Kyoto, 615-8510}
%   \location{Kyoto, JP}
  \newline
  \homepage{\href{https://fibomat.github.io/blog}{https://fibomat.github.io/blog}}
%   \twitter{@marissamayer}
%   \linkedin{linkedin.com/in/marissamayer}
  \github{\href{https://github.com/fibomat}{fibomat}} % I'm just making this up though.
%   \orcid{orcid.org/0000-0000-0000-0000} % Obviously making this up too. If you want to use this field (and also other academicons symbols), add "academicons" option to \documentclass{altacv}
}

%% Make the header extend all the way to the right, if you want.
\begin{fullwidth}
\makecvheader
\end{fullwidth}

%% Depending on your tastes, you may want to make fonts of itemize environments slightly smaller
\AtBeginEnvironment{itemize}{\small}

%% Provide the file name containing the sidebar contents as an optional parameter to \cvsection.
%% You can always just use \marginpar{...} if you do
%% not need to align the top of the contents to any
%% \cvsection title in the "main" bar.
\cvsection[page1sidebar_en]{Research}

\cvevent{Broadband frequency entangled photon generation using silicon nitride ring cavities}{with Prof. Shigeki Takeuchi}{2017/10 -- Present}{Kyoto University, Japan}
We are collaborating with Yokoyama Lab at Kyushu Univ., and focusing on realization of on-chip high-intensity broadband entangled photon sources based on SiNx and other material platforms.
\begin{itemize}
    \item study of phase matching condition for entangled photon source
    % \item design and simulation of enhanced slot-waveguides 
    \item optical device nano-fabrication, especially high quality factor micro-ring resonators
    \item realization of long-time stable frequency-correlated photon pairs broadband
\end{itemize}
\divider

\cvevent{Integrated Quantum Photonics}{with Prof. Xiaosong Ma}{2015/09 -- 2017/06}{Nanjing University, China}
My initial research interest in Malab was QKD and Bell states generation and finally turned into the integration of quantum optical devices.
\begin{itemize}
    \item nano-fabrication based on commercial SOI/SiNOI wafers
    \item demonstration and automation of the integration photonics test system
\end{itemize}
\divider

% \cvevent{Electrorheological Fluids}{with Prof. Yonghua Pan and Prof. Huibin Gao}{Nov. 2014 - Oct 2015}{}
% As a part of undergraduate training program, we evaluated several kinds of EF, and simulated the interaction with electromagnetic fields. Eventually, we succeeded to demonstrate a practical device. \divider
 

\cvsection{Education}
\cvevent{M.Eng. in electronics}{Department of Electronic Science and Engineering, Graduate School of Engineering, Kyoto University}{2017/10 -- 2020/03}{}%{Kyoto, Japan}
\divider
\cvevent{B.S. in Physics}{Department of Physics, School of Physics, Nanjing University}{2013/09 -- 2017/06}{}%{Nanjing, China}

% \cvsection{A Day of My Life}

% % Adapted from @Jake's answer from http://tex.stackexchange.com/a/82729/226
% % \wheelchart{outer radius}{inner radius}{
% % comma-separated list of value/text width/color/detail}
% % Some ad-hoc tweaking to adjust the labels so that they don't overlap
% \wheelchart{1.5cm}{0.5cm}{%
%   10/10em/accent!30/Sleeping \& dreaming about work,
%   25/9em/accent!60/Public resolving issues with Yahoo!\ investors,
%   5/13em/accent!10/\footnotesize\\[1ex]New York \& San Francisco Ballet Jawbone board member,
%   20/15em/accent!40/Spending time with family,
%   5/8em/accent!20/\footnotesize Business development for Yahoo!\ after the Verizon acquisition,
%   30/9em/accent/Showing Yahoo!\ employees that their work has meaning,
%   5/8em/accent!20/Baking cupcakes
% }


\clearpage

\cvsection[page2sidebar_en]{Conferences}

\cvevent{Design and fabrication of a silicon nitride ring resonator for on-chip broadband entangled photon sources  (oral)}{The 80th JSAP Autumn Meeting 2019, JSAP-OSA Joint Symposia}{2019/09}{Sapporo, Japan}
\cvevent{Research of CVD methods for impacts on dispersion in a SiNx ring resonator (oral \& poster)}{The 38th Electronic Materials Symposium}
{2019/10}{Nara, Japan}
\cvevent{On-chip broadband entangled photon sources using HICDG and SiN waveguide devices (poster)}{EU-USA-Japan International Symposium on Quantum Technology}
{2019/12}{Kyoto, Japan}

% \cvevent{2nd National Conference on Condensed Matter Physics}{}
% {July 2015}{Nanjing, China}
% \cvevent{12th National Seminar on Demonstrative Physics of Higher Education}{}
% {October 2015}{Suzhou, China}
% \cvevent{2014 Physics Summer School for Top Students in Zhejiang University}{}
% {July 2014}{Hangzhou, China}
 

% \cvsection{Publications}

% \nocite{*}

% \printbibliography[heading=pubtype,title={\printinfo{\faBook}{Books}},type=book]

% \divider

% \printbibliography[heading=pubtype,title={\printinfo{\faFileTextO}{Journal Articles}}, type=article]

% \divider

% \printbibliography[heading=pubtype,title={\printinfo{\faGroup}{Conference Proceedings}},type=inproceedings]

% % If the NEXT page doesn't start with a \cvsection but you'd
% % still like to add a sidebar, then use this command on THIS
% % page to add it. The optional argument lets you pull up the
% % sidebar a bit so that it looks aligned with the top of the
% % main column.
% \addnextpagesidebar[-1ex]{page3sidebar}


\end{document}
